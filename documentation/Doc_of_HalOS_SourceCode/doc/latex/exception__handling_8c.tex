\hypertarget{exception__handling_8c}{
\section{E:/Ausbildung/Semester3/Kopie von AVR32\_\-Work1/Halos\_\-Development/src/hal/ports/avr32/ap7x/ap7000/exceptions/src/exception\_\-handling.c File Reference}
\label{exception__handling_8c}\index{E:/Ausbildung/Semester3/Kopie von AVR32\_\-Work1/Halos\_\-Development/src/hal/ports/avr32/ap7x/ap7000/exceptions/src/exception\_\-handling.c@{E:/Ausbildung/Semester3/Kopie von AVR32\_\-Work1/Halos\_\-Development/src/hal/ports/avr32/ap7x/ap7000/exceptions/src/exception\_\-handling.c}}
}
NEWLIB\_\-ADDONS exception handling file for AVR32.  


{\tt \#include $<$avr32/io.h$>$}\par
{\tt \#include $<$stdio.h$>$}\par
{\tt \#include \char`\"{}../inc/exceptions.h\char`\"{}}\par
{\tt \#include \char`\"{}graphics.h\char`\"{}}\par
{\tt \#include \char`\"{}font.h\char`\"{}}\par
{\tt \#include \char`\"{}halosUtil.h\char`\"{}}\par
\subsection*{Functions}
\begin{CompactItemize}
\item 
void \hyperlink{exception__handling_8c_eb0fee7e25692380f5346f304f4b6e5a}{\_\-\_\-attribute\_\-\_\-} ((weak))
\end{CompactItemize}


\subsection{Detailed Description}
NEWLIB\_\-ADDONS exception handling file for AVR32. 

\begin{itemize}
\item Compiler: GNU GCC for AVR32\item Supported devices: All AVR32 devices can be used.\item AppNote:\end{itemize}


\begin{Desc}
\item[Author:]Atmel Corporation: \href{http://www.atmel.com}{\tt http://www.atmel.com} \par
 Support and FAQ: \href{http://support.atmel.no/}{\tt http://support.atmel.no/} \end{Desc}


\subsection{Function Documentation}
\hypertarget{exception__handling_8c_eb0fee7e25692380f5346f304f4b6e5a}{
\index{exception\_\-handling.c@{exception\_\-handling.c}!\_\-\_\-attribute\_\-\_\-@{\_\-\_\-attribute\_\-\_\-}}
\index{\_\-\_\-attribute\_\-\_\-@{\_\-\_\-attribute\_\-\_\-}!exception_handling.c@{exception\_\-handling.c}}
\subsubsection[{\_\-\_\-attribute\_\-\_\-}]{\setlength{\rightskip}{0pt plus 5cm}int \_\-\_\-attribute\_\-\_\- ((weak))}}
\label{exception__handling_8c_eb0fee7e25692380f5346f304f4b6e5a}


Low-level write command. When newlib buffer is full or fflush is called, this will output data to correct location. 1 and 2 is stdout and stderr which goes to usart 3 is framebuffer 