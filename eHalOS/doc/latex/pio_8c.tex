\hypertarget{pio_8c}{
\section{E:/Ausbildung/Semester3/AVR32\_\-Work1/Halos\_\-Development/src/hal/ports/avr32/ap7x/ap7000/pio.c File Reference}
\label{pio_8c}\index{E:/Ausbildung/Semester3/AVR32\_\-Work1/Halos\_\-Development/src/hal/ports/avr32/ap7x/ap7000/pio.c@{E:/Ausbildung/Semester3/AVR32\_\-Work1/Halos\_\-Development/src/hal/ports/avr32/ap7x/ap7000/pio.c}}
}
pio driver functions  


{\tt \#include \char`\"{}pio.h\char`\"{}}\par
\subsection*{Functions}
\begin{CompactItemize}
\item 
volatile avr32\_\-pio\_\-t $\ast$ \hyperlink{pio_8c_0bd74eb36ce8ba84b30411571192a207}{pioGetHandle} (int port)
\item 
int \hyperlink{pio_8c_11350a4ffdfbbd00abb0acee8faefcfa}{pio\_\-setup\_\-pin} (int pin, int function)
\item 
void \hyperlink{pio_8c_45c2e62c9da21ea0aea60d9dcd8023fa}{pio\_\-enable\_\-module} (\hyperlink{pio_8h_e3e4e5c4f0784fd66613f9aedffb758a}{avr32\_\-piomap\_\-t} piomap, int size)
\end{CompactItemize}


\subsection{Detailed Description}
pio driver functions 

This file contains basic pio driver functions.

\begin{itemize}
\item Compiler: IAR EWAAVR32 and GNU GCC for AVR32\item Supported actDevices: All AVR32 actDevices with a PIO module can be used. The example is written for AP7000 and STK1000.\item AppNote: AVR32111 - Using the AVR32 PIO Controller\end{itemize}


\begin{Desc}
\item[Author:]Atmel Corporation: \href{http://www.atmel.com}{\tt http://www.atmel.com} \par
 Support email: \href{mailto:avr32@atmel.com}{\tt avr32@atmel.com}\end{Desc}
\begin{Desc}
\item[Name]\end{Desc}
\begin{Desc}
\item[Revision]1.4 \end{Desc}
\begin{Desc}
\item[RCSfile]\hyperlink{pio_8c}{pio.c},v \end{Desc}
\begin{Desc}
\item[Date]2006/05/10 12:01:41 \end{Desc}


\subsection{Function Documentation}
\hypertarget{pio_8c_45c2e62c9da21ea0aea60d9dcd8023fa}{
\index{pio.c@{pio.c}!pio\_\-enable\_\-module@{pio\_\-enable\_\-module}}
\index{pio\_\-enable\_\-module@{pio\_\-enable\_\-module}!pio.c@{pio.c}}
\subsubsection[{pio\_\-enable\_\-module}]{\setlength{\rightskip}{0pt plus 5cm}void pio\_\-enable\_\-module ({\bf avr32\_\-piomap\_\-t} {\em piomap}, \/  int {\em size})}}
\label{pio_8c_45c2e62c9da21ea0aea60d9dcd8023fa}


This function will enable a module pin for a given set of pins and respective modules \begin{Desc}
\item[Parameters:]
\begin{description}
\item[{\em $\ast$piomap}]A list of pins and pio connectivity \item[{\em size}]The number of pins in $\ast$piomap \end{description}
\end{Desc}
\begin{Desc}
\item[Returns:]nothing \end{Desc}
\hypertarget{pio_8c_11350a4ffdfbbd00abb0acee8faefcfa}{
\index{pio.c@{pio.c}!pio\_\-setup\_\-pin@{pio\_\-setup\_\-pin}}
\index{pio\_\-setup\_\-pin@{pio\_\-setup\_\-pin}!pio.c@{pio.c}}
\subsubsection[{pio\_\-setup\_\-pin}]{\setlength{\rightskip}{0pt plus 5cm}int pio\_\-setup\_\-pin (int {\em pin}, \/  int {\em function})}}
\label{pio_8c_11350a4ffdfbbd00abb0acee8faefcfa}


This function will put a single pin under a module's control \begin{Desc}
\item[Parameters:]
\begin{description}
\item[{\em $\ast$pin}]The pin number \item[{\em $\ast$function}]The PIO module which to enable \end{description}
\end{Desc}
\begin{Desc}
\item[Returns:]PIO\_\-SUCCESS or PIO\_\-INVALID\_\-ARGUMENT \end{Desc}
\hypertarget{pio_8c_0bd74eb36ce8ba84b30411571192a207}{
\index{pio.c@{pio.c}!pioGetHandle@{pioGetHandle}}
\index{pioGetHandle@{pioGetHandle}!pio.c@{pio.c}}
\subsubsection[{pioGetHandle}]{\setlength{\rightskip}{0pt plus 5cm}volatile avr32\_\-pio\_\-t$\ast$ pioGetHandle (int {\em port})}}
\label{pio_8c_0bd74eb36ce8ba84b30411571192a207}


This function will return the baseaddress for a port \begin{Desc}
\item[Parameters:]
\begin{description}
\item[{\em $\ast$port}]The port number \end{description}
\end{Desc}
\begin{Desc}
\item[Returns:]The port's baseaddress \end{Desc}
